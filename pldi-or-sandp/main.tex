\newif\ifsandp
%\sandptrue % => S&P
\sandpfalse % => PLDI

\ifsandp
\documentclass[letterpaper,10pt]{IEEEtran} % 15 + 5 pages
\else
\documentclass[pldi,10pt]{sigplanconf-pldi15} % 11 + \infty (refs only) pages
\fi
%\pagestyle{empty}

\usepackage{paralist}
\usepackage{color}

\title{Verifying Constant-Time Properties of Low-Level Software}
\ifsandp\author{\color{red}%%
Deadline: November 13th, 23:59 EST (4:59 CET)\\ %%
Page limit: 15 pages + 5 for refs and appendices}
%%
\else\subtitle{\color{red}%%
Deadline: November 20th, 18:00 EST (23:00 CET);\\ %%
11 pages + anything we need for refs only}
%%
\fi

\begin{document}

\maketitle

\section{Introduction}

General motivation regarding the importance of side-channel
countermeasures (timing and cache-timing in particular).

\subsection{Motivation - Specific to Constant-Time}
Constant-time-specific motivation:
\begin{compactitem}
\item lack of formal justification for balancing,
\item difficulty of verifying/validating adherence to complex
  policies;
\item importance of proving this at a low level.
\end{compactitem}

\subsection{Contributions}
\begin{compactitem}
\item Attack (Downplay for PLDI? Keep central for S\&P?)
\item Definitional: formalization of constant-time (as a non-interference 
property and for procedure calls)
\item Theoretical result: complete and sound product program
  construction.
\item Theoretical result: leads for declassification.
\item Practical result: first constant-time verification for many
  examples, some (important?) bugs found;
\item Practical result: flexibility w.r.t. leakage model: any
  \emph{compositional} leakage model stronger than or equal to Program
  Counter. (why compositional?)
\end{compactitem}

\section{Definitions and discussion}

\begin{compactitem}
\item Define constant-time on LLVM-like language
\item Discuss source vs assembly
\item Instructions that depend on operand size and other limitations
\item Quantitative analysis might be deceiptive (weak model)
\item Constant-time provides guarantees in strond adversary model
\end{compactitem}

\section{Product Programs for Constant-Time Verification}

\begin{compactitem}
\item Describe construction;
\item Prove soundess + completeness
\item Variants: size-dependent execution time (DIV) on a small example
\end{compactitem}

\section{Constant-Time with Declassified Outputs}

\begin{compactitem}
\item Justify usefulness of declassification;
\item Describe construction;
\item Prove soundness;
\item Prove restricted completeness: single declassified branch that
  is not inside a loop is ok (unsure)
\end{compactitem}

\section{Implementation}

\begin{compactitem}
\item Outline the toolchain;
\item Outline the product program construction (easy);
\item Outline the challenges to overcome in front and back-end in
  order to get practical results out of the toolchain:
  \begin{compactitem}
  \item Front-end robustness and precision;
  \item Scalability of back-end: loop unrolling not possible in some
    cases.
  \end{compactitem}
\item Outline our solution choices and their limitations:
  \begin{compactitem}
  \item DSA for memory separation; use a robust front-end;
  \item Loop invariant and procedure contract inference: the easy
    analysis is enough for most examples! (How do we choose whether to
    inline or modularize?)
  \item In case it is not, there is a manual fallback (at source level
    or back-end level?).
  \end{compactitem}
\end{compactitem}

\section{Experimental results}

\begin{compactitem}
\item PolarSSL: DES-CBC, SHA256, MEE-CBC(? -- need to check function pointers);
\item FTFP: all functions -- fix the \texttt{fixfrac} function to be
  constant-time;
\item RLWE (Stebila et al.) -- sampling for now. Others should be
  possible;
\item mee-cbc (ours): simple + declassification (+ leaky integer division?)
\item pkcs (ours): not sure if we can do anything yet;
\item curve25519-donna: done (cut at call graph choke points)
\item libsodium: ??
\item TEA: done
\item ideally: Salsa, ChaCha (maybe already in libsodium)
\item some negative examples
\end{compactitem}

\section{Related work}

\begin{compactitem}
\item Important to point out: first static analysis below C that scales somewhat;
\item Important to point out: no marking secrets, mark public data
  instead (failure by default seems important to some);
\item Support for declassification is important step for precision
  (hence adoption). We are first to even consider it.
\end{compactitem}

\section{Conclusions and directions for future work}



% Biblio
\bibliographystyle{plain}

% Appendices
\appendix

\begin{compactitem}
\item Details of the attack?
\item Details of some negative examples?
\item Details of some landmark positive examples?
\end{compactitem}

\end{document}
